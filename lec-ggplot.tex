% grep -o -h -r "geom_[^(]*(" . --include="*R" | sort | uniq
\documentclass[10pt]{beamer}

\usetheme{Singapore} %-wolverine}
\usecolortheme{seahorse}


    \setbeamertemplate{headline}
    {%
%       \begin{beamercolorbox}[ht=3.5ex,dp=1.125ex,%
%           leftskip=.3cm,rightskip=.3cm plus1fil]{section in head/foot}
%         \usebeamerfont{section in head/foot}\usebeamercolor[fg]{section in head/foot}%
%         \insertsectionhead
 %      \end{beamercolorbox}%
       \begin{beamercolorbox}[colsep=1.5pt]{middle separation line head}
       \end{beamercolorbox}
       \begin{beamercolorbox}[ht=2.5ex,dp=1.125ex,%
         leftskip=.3cm,rightskip=.3cm plus1fil]{subsection in head/foot}
         \usebeamerfont{subsection in head/foot}\insertsubsectionhead
       \end{beamercolorbox}%
       \begin{beamercolorbox}[colsep=1.5pt]{lower separation line head}
       \end{beamercolorbox}
    }
    
\usepackage{color}
\usepackage{times}
\usepackage{hyperref}
\usepackage{tikz}
\renewcommand{\arraystretch}{1.5}

\newcommand{\meanw}{\langle w \rangle}
\newcommand{\meanm}{\langle m \rangle}
\newcommand{\vw}{\langle \xi_w^2 \rangle}
\newcommand{\vm}{\langle \xi_m^2 \rangle}
\newcommand{\vwm}{\langle \xi_w \xi_m \rangle}
\newcommand{\meanh}{\langle h \rangle}
\newcommand{\vh}{\langle h^2 \rangle}
\newcommand{\eh}{\langle h \rangle}
\newcommand{\vwp}{\langle w^2 \rangle}
\newcommand{\vmp}{\langle m^2 \rangle}
\newcommand{\meanwm}{\langle w + m \rangle}
\newcommand{\vwmp}{\langle (w + m)^2 \rangle}
\newcommand{\cmwmp}{\langle m (w + m) \rangle}
\newcommand{\cwm}{\langle w m \rangle}

\newcommand{\mycode}[1]{\texttt{#1}}
\newcommand{\mycodex}[1]{\texttt{\textcolor{red}{#1}}}
\newcommand{\mycite}[1]{\small{[\textcolor{gray}{#1}]} \normalsize}
\newcommand{\mynote}[1]{\textcolor{red}{(#1)}}
\newcommand{\myanswer}[1]{\textcolor{red}{ $\rightarrow$ #1}}
\newcommand{\figsrc}[1]{\\ \textcolor{gray}{\tiny{Figure source: #1}}}

\newcommand{\myvec}[1]{\underline{#1}}
\newcommand{\mymat}[1]{\underline{\underline{#1}}}
\newcommand{\iaindu}[1]{\mymat{#1}}
\newcommand{\iainsu}[1]{\myvec{#1}}

\newcommand{\igjlight}{\textcolor{red}{\textbf{$\bigstar$}}}
\newcommand{\igjstar}{\textcolor{red}{\textbf{$\bigstar$}}}

\newcommand{\myglossary}{\textcolor{gray}{\tiny{($w,m$ wildtype, mutant copy nums; $h = m/(w+m)$, $\lambda_i, \beta_i$ replication; $\nu_i, \delta_i$ degradation; $\phi_i$ deterministic parts; $\xi_i$ fluctuation parts)}} }

\newcommand\importantstuff[3]{
  \node[black!20!white] at (#1+0.08,#2-0.08) {
    \scalebox{2}{\large{#3}}
  };
  \node at (#1,#2) {
    \scalebox{2}{\large{#3}}
  };
}

\newenvironment{myitemize}
{ \begin{itemize}
    \setlength{\itemsep}{0pt}
    \setlength{\parskip}{0pt}
    \setlength{\parsep}{0pt}     }
{ \end{itemize}                  } 

\definecolor{light-gray}{gray}{0.95}

%\usetheme{Darmstadt}
%\usefonttheme[onlylarge]{structurebold}
%\setbeamerfont*{frametitle}{size=\normalsize,series=\bfseries}
%\setbeamertemplate{navigation symbols}{}

% \usecolortheme{crane}
 % \beamertemplatesolidbackgroundcolor{craneorange!25}
\setbeamerfont{frametitle}{size=\normalsize}

\makeatletter
  \newcommand\tinyv{\@setfontsize\tinyv{1pt}{6}}
\makeatother

\newcommand{\figcred}[1]{\vspace{0.2cm} \footnotesize{\textcolor{gray}{(#1)}} \normalsize}
\newcommand{\citecred}[1]{\tiny{\textcolor{gray}{(#1)}} \normalsize}
\newcommand{\bneck}{eLife \textbf{4} e07464 (2015)}
\newcommand{\cellsys}{Cell Systems \textbf{2} 101 (2016)}
  \newcommand{\cellrep}{Cell Reports \textbf{7} 2031 (2014)}
  \newcommand{\ajhg}{IGJ \& Jones, Am J Hum Genet \textbf{99} 1150 (2016)}
  \newcommand{\sagamb}{Stat Appl Genet Mol Biol \textbf{13} 379 (2014)}
  \newcommand{\natcomms}{Nat Comms \textbf{9} 2488 (2018)}
  \newcommand{\cfour}{eLife \textbf{2} e00961 (2014)}
  \newcommand{\molplant}{Mol Plant (online 2019)}
    \newcommand{\prsa}{Proc Roy Soc A \textbf{471} 20150050 (2015)}
    \newcommand{\ogdraw}{OGDraw}
    \newcommand{\plosbiol}{das Neves et al., PLoS Biol \textbf{8} e1000560 (2010)}
    \newcommand{\kerr}{Kerr et al., submitted (2019)}
        \newcommand{\hanne}{Hoitzing et al., PLoS Comput Biol (online 2019)}
            \newcommand{\significance}{IGJ, Significance \textbf{9} 17 (2012)}
    \newcommand{\ploscb}{IGJ et al., PLoS Comput Biol \textbf{8} e1002416 (2012)}
        \newcommand{\imamura}{Yaginuma et al., Sci Rep \textbf{4} 6522 (2014)}
\makeatletter

\newcommand*\bigcdot{\mathpalette\bigcdot@{.5}}
\newcommand*\bigcdot@[2]{\mathbin{\vcenter{\hbox{\scalebox{#2}{$\m@th#1\bullet$}}}}}
\makeatother


%\title{Evolutionary inference reveals mtDNA gene retention dependence on hydrophobicity, GC content, and control of complex assembly}
\title{Visualisation with \texttt{ggplot2}: A primer}
\institute{\vspace{-0.8cm} \\ \large Iain Johnston \vspace{0.25cm} \\ \scriptsize Department of Mathematics\\ University of Bergen\\ \vspace{0.4cm}} %\includegraphics[width=0.6\textwidth]{funders.png} }
\date{}
%\date{}

% The main document

\begin{document}

\frame{\titlepage}

\begin{frame}{Hello!}
  \begin{columns}
    \begin{column}{0.5\textwidth}
      Disclaimer: I have no formal training in data visualization and everything here will be principles I work to and approaches I use, rather than a rigorous and complete syllabus on data visualization theory and practice. For that (and ECTS credits) look to \url{https://vis.uib.no/teaching/}.
    \end{column}
    \begin{column}{0.5\textwidth}
      \includegraphics[width=\linewidth]{ex-published.png}
    \end{column}
    \end{columns}
\end{frame}

\begin{frame}{Why do we produce plots?}
\end{frame}

\begin{frame}{Why do we produce plots?}
      \begin{itemize}
      \item To summarise and convey data
      \item To turn data into information
      \item To tell a story
        \item[]
      \end{itemize}
      \includegraphics[width=\linewidth]{ex-1.png}
\end{frame}

\begin{frame}{Why do we produce plots?}
      \begin{itemize}
      \item To summarise and convey data
      \item To turn data into information
      \item To tell a story
        \item ... clearly, transparently, honestly, reproducably, beautifully, pre-empting questions
      \end{itemize}
      \includegraphics[width=\linewidth]{ex-2.png}
\end{frame}

\begin{frame}{Why do we produce plots?}
      \begin{itemize}
      \item `Friends don't let friends' (Chenxin Li) \url{https://github.com/cxli233/FriendsDontLetFriends}
        \item Complete ggplot tutorial \url{http://r-statistics.co/Complete-Ggplot2-Tutorial-Part1-With-R-Code.html}
      \end{itemize}
\end{frame}


\begin{frame}{Basic principles: aesthetic}
    \begin{itemize}[<+->]
    \item \textbf{Motivation:} A reader will, at first, only spend a couple of seconds glancing at your figure.
      \item That's whether they're skimming an article, looking around at posters, or looking at your slide while they're watching you talk
      \item Even if they spend longer, the easier the messages come through, the better
      \item \textbf{Therefore:} The big picture should be clear and get the most important message, and as much transparent support as you can provide, across at a glance
        \item It should also look cool! Especially if your reader's attention is tempted elsewhere (it is)
      \item A figure is not just an image! In written work, the caption is essential too (art museums). In oral work, your description plays this role.
      \end{itemize}
\end{frame}


\begin{frame}{Basic principles: aesthetic}
    \begin{itemize}
    \item Match scales, colours, fonts (basically, consistent style)
    \item Have information occupy most of the plot
    \item Plot as close to individual datapoints as possible
    \item Every approximation (especially smoothing) needs explicit mention
    \item Define all degrees of freedom (only) in figure caption / oral commentary
      \item Consider accessibility
  \end{itemize}
\end{frame}


\begin{frame}{Basic principles: aesthetic}
    \begin{itemize}
    \item Match scales, colours, fonts (basically, consistent style)
    \end{itemize}
    \includegraphics[width=\textwidth]{bad-1.png}
\end{frame}
\begin{frame}{Basic principles: aesthetic}
    \begin{itemize}
    \item Match scales, colours, fonts (basically, consistent style)
    \end{itemize}
    \includegraphics[width=.4\textwidth]{peppa1.jpg} \\
    \figcred{Peppa goes to London}
\end{frame}
\begin{frame}{Basic principles: aesthetic}
    \begin{itemize}
    \item Match scales, colours, fonts (basically, consistent style)
    \end{itemize}
    \includegraphics[width=.4\textwidth]{peppa2.jpg} \\
        \figcred{Peppa goes to London}
\end{frame}\begin{frame}{Basic principles: aesthetic}
    \begin{itemize}
    \item Match scales, colours, fonts (basically, consistent style)
    \end{itemize}
    \includegraphics[width=.6\textwidth]{peppa3.jpg} \\
        \figcred{Peppa goes to London}
\end{frame}
\begin{frame}{Basic principles: aesthetic}
    \begin{itemize}
    \item Match scales, colours, fonts (basically, consistent style)
    \end{itemize}
    \centering
        \includegraphics[width=.6\textwidth]{encode.png} \\
    \includegraphics[width=.6\textwidth]{serum_project_logo.jpg} \\
    \figcred{\url{https://www.cshl.edu/encode3-interpreting-the-human-and-mouse-genomes/}, \url{https://www.nist.gov/image/serum-project-logo}}
\end{frame}

\begin{frame}{Basic principles: aesthetic}
  \begin{columns}
    \begin{column}{0.6\textwidth}
    \begin{itemize}
    \item Fonts: just use Arial (or Helvetica, though it is slightly more embellished)
    \item Serifs are for guiding the eye through large blocks of text; they complicate text in small sections
    \item Myriad Pro (Illustrator default) looks like we're learning letters in primary school
      \item DejaVu Sans (matplotlib default) looks like the font designer is still learning letters (and kerning) in primary school
    \end{itemize}
    \end{column}
    \begin{column}{0.6\textwidth}
      \includegraphics[width=\linewidth]{sans-serifs.png}
    \end{column}
    \end{columns}
\end{frame}

\begin{frame}{Basic principles: aesthetic}
    \begin{itemize}
    \item Have information occupy most of the plot
    \end{itemize}
    \includegraphics[width=\textwidth]{bad-2.png}
\end{frame}

\begin{frame}{Basic principles: aesthetic}
    \begin{itemize}
    \item Plot as close to individual datapoints as possible
    \end{itemize}
    \includegraphics[width=\textwidth]{bad-3.png}
\end{frame}
\begin{frame}{Basic principles: aesthetic}
    \begin{itemize}
    \item Plot as close to individual datapoints as possible
    \item Showing more data is better; showing less data is worse
  \item If you want to summarise, why not do both?
    \end{itemize}
 \centering
    \includegraphics[width=0.6\linewidth]{sems.jpg}
 
 \mycite{\emph{Common Misconceptions about Data Analysis and Statistics}, Motulsky, J Pharm Expt Therap 351 200 (2014)}
\end{frame}

\begin{frame}{Basic principles: aesthetic}
    \begin{itemize}
    \item Every approximation (especially smoothing) needs explicit mention. What parameters did you use? (Without knowing?)
    \end{itemize}
    \includegraphics[width=\textwidth]{bad-4.png}
\end{frame}
\begin{frame}{Basic principles: aesthetic}
    \begin{itemize}
    \item Every approximation (especially smoothing) needs explicit mention. What parameters did you use? (Without knowing?)
    \end{itemize}
    \includegraphics[width=\textwidth]{bad-5.png}
\end{frame}
\begin{frame}{Basic principles: aesthetic}
    \begin{itemize}
    \item Every approximation (especially smoothing) needs explicit mention. What parameters did you use? (Without knowing?)
    \end{itemize}
    \includegraphics[width=\textwidth]{bad-6.png}
\end{frame}

\begin{frame}{Basic principles: aesthetic}
    \begin{itemize}
    \item Define all degrees of freedom (only) in figure caption / oral commentary
    \end{itemize}
    \includegraphics[width=\textwidth]{text-embedding.png}
\end{frame}

\begin{frame}{Basic principles: aesthetic}
    \begin{itemize}
    \item Define all degrees of freedom (only) in figure caption / oral commentary
      \item Information flow: the figure (image + caption) must give the message on its own -- it's all that some people will read. That message can then be `cited' in the main text
    \end{itemize}
    \centering
    \includegraphics[width=.6\textwidth]{text-embedding.png}
\end{frame}


\begin{frame}{Basic principles: production}
    \begin{itemize}
    \item We'll use R and \texttt{ggplot2} (and additions)
      \item Part of the \texttt{tidyverse}
      \item Strongest for 2D plots; for 3D, consider looking elsewhere
            \item Aim to make all code and data for a project public, and to have a single script that produces all the display items for the paper
            \item FAIR reporting; but this also helps investigation!
              
    \end{itemize}
    \includegraphics[width=.7\textwidth]{fair.png}
\end{frame}

\begin{frame}{Basic principles: production}
    \begin{itemize}[<+->]
      \item We need a file output for basically all purposes except live production
    \item Bitmap/raster (JPG, TIFF, PNG, GIF, PSD) vs vector (PS, EPS, PDF, AI, SVG) formats
    \item I recommend PNG\footnote{JPG loses info; TIFF compression hard to implement in R} and SVG(lite)
    \item Have your script get as close to published figure as possible (\texttt{ggpubr} for panels, annotation etc)
    \item One image file per figure
      \item For bitmaps, fix image size (in pixels), although journals will require 300dpi. 2490 pixels is A4 width at 300dpi
  \end{itemize}
\end{frame}

\begin{frame}{Basic principles: production}
    \begin{itemize}
    \item My approach:
    \item[] \texttt{\textcolor{red}{png}("filename.png", width=\textcolor{blue}{400}*\textcolor{orange}{sf}, height=\textcolor{blue}{300}*\textcolor{orange}{sf}, res=72*\textcolor{orange}{sf})}
      \item[] \texttt{print(} (plot command) \texttt{)}
      \item[] \texttt{dev.off()}
      \item \texttt{\textcolor{red}{png}} sets output `terminal' to PNG file
      \item \textcolor{blue}{400 $\times$ 300} gives aspect ratio and scale of graphical elements; adjust to style output. Don't rescale in post-processing!
      \item \texttt{\textcolor{orange}{sf}} scales resolution, \emph{keeping those features intact}
        \item Also consider \texttt{ggsave}
  \end{itemize}
\end{frame}


\begin{frame}{Basic principles: input data for \texttt{ggplot2}}
  \begin{itemize}
        \item What are the types of variables to plot? Continuous, discrete, factor, other? This determines plot types
    \item Long (not wide) data frames
      \item Every row is a datapoint; (possibly many) columns describe predictors and labels
    \item \texttt{reshape2} and \texttt{dplyr} offer functions to convert (pivot, melt, reshape)
    \end{itemize}
    \includegraphics[width=.55\textwidth]{wide-data.png}     \includegraphics[width=.4\textwidth]{long-data.png} \\
    \figcred{https://www.rforecology.com/post/reshaping-data-in-r/}
\end{frame}

\begin{frame}{Basic principles: \texttt{ggplot2}}
    \begin{itemize}
    \item `Grammar of graphics' (Leland Wilkinson, 2005)
    \item Idea: build up graphics using system of elements (data, geometric objects, styling features, ...) combined using given rules
    \item Effectively: the final plot is an object which is built by adding the output of different functions
    \item Each function controls some elements of the grammar, specifying the details of our plot
      \item We'll look at a subset. Online searches are your (and my!) friend for completeness (remember to cite sources though!)
    \end{itemize}
\end{frame}

\begin{frame}{Basic principles: \texttt{ggplot2}}
    \begin{itemize}
    \item \textcolor{red}{Plot command} + \textcolor{blue}{data} + \textcolor{brown}{mapping/aesthetic} + \textcolor{purple}{geoms} + \textcolor{gray}{styling options}
    \item \texttt{\textcolor{red}{ggplot(}\textcolor{blue}{data = my.df,} \textcolor{brown}{mapping = aes(x=x, y=y)}\textcolor{red}{)} \textcolor{purple}{+ geom\_point()} \textcolor{gray}{+ theme\_light()}}
    \item Implicit: \texttt{\textcolor{red}{ggplot(}\textcolor{blue}{my.df,} \textcolor{brown}{aes(x=x, y=y)}\textcolor{red}{)} \textcolor{purple}{+ geom\_point()}}\textcolor{gray}{\texttt{+ theme\_light()}}
    \item \textcolor{red}{Plot command} + \textcolor{blue}{data} hopefully intuitive
    \item \textcolor{brown}{mapping/aesthetic} defines which columns in the data correspond to which aspects of the plot
    \item \textcolor{purple}{geoms} define which types of plot we're producing and specify overall presentational aspects for the data
      \item \textcolor{gray}{styling options} set presentational aspects additional to the data (axes, frame, fonts, etc)
        \end{itemize}
\end{frame}

\begin{frame}{Basic principles: \texttt{ggplot2}}
  \begin{itemize}
  \item Let's try it!
    \item[] \texttt{library(ggplot2)}
    \item[] \texttt{my.df = data.frame(x=1:10, y=1:10+rnorm(10))}
    \item[] \texttt{\textcolor{red}{ggplot(}\textcolor{blue}{my.df,} \textcolor{brown}{aes(x=x, y=y)}\textcolor{red}{)} \textcolor{purple}{+ geom\_point()}}\textcolor{gray}{\texttt{+ theme\_light()}}
         \end{itemize}
\end{frame}

\begin{frame}{Data styling}
    \begin{itemize}
    \item Geoms have a set of `attributes' which may be connected to data (via \texttt{aes}) or specified otherwise
      \item[] e.g. \texttt{geom\_point(aes(x=x,y=y,colour=group), size=2, alpha=0.5)}
    \item Common ones:
      \begin{itemize}
      \item \texttt{size}
      \item \texttt{fill}
      \item \texttt{colour}
      \item \texttt{alpha}
      \item \texttt{shape}
      \end{itemize}
      \item Size is relative (to dimensions of plot) -- this will change for different output size/resolution combinations!
      \item Colour can be specified in several ways: some names (\texttt{red}, \texttt{green} etc), \texttt{\#RRGGBB}, \texttt{\#RRGGBBAA} 
    \end{itemize}
\end{frame}

\begin{frame}{Exercise}
  \begin{itemize}
  \item Continuous-continuous data
    \item[] \texttt{data(mtcars)}
    \item \texttt{mtcars} has data on several car models. \texttt{mpg} (miles per gallon), \texttt{disp} displacement, \texttt{cyl} number of cylinders
     \item Graphically explore the relationship between these. Can you reproduce these plots?
  \end{itemize}
  \includegraphics[width=\linewidth]{exercise-cars.png}
\end{frame}

\begin{frame}{\texttt{geom\_...} for continuous-continuous}
    \begin{itemize}
    \item \texttt{point}: points
    \item \texttt{line}: lines (connecting points by $x$)
       \item \texttt{path}: path (connecting point by order in data)
        \item \texttt{segment}: line segments (specifying start and end points \texttt{xend, yend})
       \item \texttt{smooth}: smoothed lines, including regression (\texttt{method=''lm''})
             \item \texttt{tile}: filled tiles (specifying fill colour \texttt{fill}); also \texttt{rect}
               \item \texttt{text}, \texttt{label}, \texttt{text\_repel}*: text (specifying text \texttt{label}) with and without box, and avoiding overlaps
                 \item \texttt{bin\_2d}: density of points in 2D space
               \end{itemize}
\end{frame}


\begin{frame}{\texttt{geom\_...} for continuous-continuous}
  \includegraphics[width=\textwidth]{cont-cont.png}
\end{frame}

\begin{frame}{\texttt{geom\_...} for factor-continuous}
    \begin{itemize}
    \item \texttt{point}: points
      \item \texttt{col}: columns
    \item \texttt{jitter} and \texttt{beeswarm}*: points arranged to avoid overlaps
       \item \texttt{boxplot}: box plots
             \item \texttt{violin}: violin plots
    \item \texttt{histogram}: histogram (consider \texttt{bins})
              \end{itemize}
\end{frame}

\begin{frame}{\texttt{geom\_...} for factor-continuous}
  \includegraphics[width=\textwidth]{fact-cont.png}
\end{frame}

\begin{frame}{Exercise}
  \begin{itemize}
  \item Factor-continuous data
    \item[] \texttt{data(PlantGrowth)}
    \item \texttt{PlantGrowth} has data on plant yield under different treatments (and control). \texttt{weight} is yield (dry weight), \texttt{group} is treatment
     \item Graphically explore the relationship between these. Can you reproduce this plot?
  \end{itemize}
  \includegraphics[width=.5\linewidth]{exercise-growth.png}
\end{frame}

\begin{frame}{\texttt{geom\_...} for continuous-continuous-continuous}
  \begin{itemize}
  \item Takes $x,y$ scatter and works with point density:
    \begin{itemize}
                     \item \texttt{bin\_2d}: filled rectangles
                     \item \texttt{hex}: filled hexagons
                       \item \texttt{geom\_density\_2d}, \texttt{geom\_density\_2d\_filled}: contours (after smoothing)
    \end{itemize}
  \item Takes $x,y,z$ data (can be produced from above by smoothing):
    \begin{itemize}
      \item \texttt{tile} and \texttt{rect} as before
                 \item \texttt{contour}, \texttt{label\_contour}*: contour plots
    \end{itemize}
    \end{itemize}
\end{frame}

\begin{frame}{\texttt{geom\_...} for continuous-continuous-continuous}
  \includegraphics[width=\textwidth]{cont-cont-z.png}
\end{frame}

\begin{frame}{Plot theme styling: arrangements}
    \begin{itemize}
    \item Organise a set of plots in 1D: \texttt{facet\_wrap($\sim$ variable)}
      \item Organise a set of plots in 2D: \texttt{facet\_grid(var1 $\sim$ var2)}
      \item Organise generally: \texttt{ggarrange(g.1, g.2, }...\texttt{)}
        \item For example, \texttt{\textcolor{blue}{ggarrange(g.1, }\textcolor{red}{ggarrange(g.2, g.3, labels=c(``B'', ``C''), nrow=2)}\textcolor{blue}{, labels=c(``A'', ``''), widths=c(1,2), nrow=1)}}
    \end{itemize}
    \includegraphics[width=\textwidth]{arrangement.png}
\end{frame}

\begin{frame}{Which geoms?}
  \includegraphics[width=\linewidth]{ggplot-examples.png}
  \end{frame}

\begin{frame}{Which geoms?}
  \includegraphics[width=\linewidth]{ggplot-advanced.png}
  \end{frame}

\begin{frame}{Plot theme styling}
    \begin{itemize}
    \item Every part of the plot is addressable as an element
    \item Overall \texttt{theme\_}...\texttt{()}: \texttt{light}, \texttt{minimal}, \texttt{bw}, \texttt{classic}
    \item \texttt{theme(}...\texttt{)}:
      \begin{itemize}
      \item \texttt{legend.position = ``none''}
      \item \texttt{axis.text.x = element\_text(size=12, angle=45, hjust=1)}
      \item \texttt{axis.line.x = element\_line(color = ``\#AAAAAA'')}
        \item \texttt{axis.title.x = element\_text(size=24,color = ``\#AAAAAA'')}
      \end{itemize}
      \end{itemize}
\end{frame}

\begin{frame}{Plot theme styling}
  \includegraphics[width=.9\textwidth]{theme_elements.png} \\
  \figcred{Henry Wang, henrywang.nl}
  \end{frame}

\begin{frame}{Plot theme styling: axes}
    \begin{itemize}
    \item Some useful tweaks:
    \item \texttt{labs(x=``x label'', y=``y label'', color=``colour scale label'', fill=``fill scale label''}, ...\texttt{)}
    \item \texttt{xlim(0, 100)}, \texttt{ylim(0, NA)} (also \texttt{lims})
    \item \texttt{scale\_x\_continuous(trans = ``log'')} -- log (or other) transformation
    \item \texttt{scale\_x\_continuous(expand = c(0.1, 0.1))} -- expand axis (e.g. to fit labels)
           \item \texttt{scale\_x\_continuous(breaks = c(0, 1, 2, 3), labels=c("Reference", "Early", "Late", "Complete"))} -- custom ticks
           \item \texttt{theme(axis.text.x = element\_text(angle=45, hjust=1))} -- diagonal tick labels
           \item For discrete scales, factor levels will be plotted in their internal order. (Re)set with
             \item[] \texttt{f = factor(f, levels=c(``First level'', ``Second level''))}
      \end{itemize}
\end{frame}

\begin{frame}{Exercise}
  \begin{itemize}
  \item Continuous-continuous data
    \item[] \texttt{data(mtcars)}
    \item \texttt{mtcars} has data on several car models. \texttt{mpg} (miles per gallon), \texttt{disp} displacement, \texttt{cyl} number of cylinders
     \item Graphically explore the relationship between these. Can you reproduce this plot?
  \end{itemize}
  \includegraphics[width=.5\linewidth]{exercise-cars2.png}
\end{frame}

\begin{frame}{Colours}
   \begin{itemize}
   \item We think differently about continuous and discrete scales
   \item For both, we can tell ggplot a set of colours, or a scheme, that defines a scale
     \item \texttt{scale\_fill\_gradient}, \texttt{gradient2}, \texttt{gradientn}: specify low-high, low-mid-high, custom colours
     \item \texttt{scale\_fill\_viridis}*: specify option (default, inferno, ...)
     \item Replace \texttt{fill} with \texttt{colour} as required
     \item For factor variables with viridis, use \texttt{discrete=TRUE}
       \item I like viridis; also consider ColorBrewer
     \end{itemize}
\end{frame}

\begin{frame}{Colours}
  \includegraphics[width=\linewidth]{ggplot-colours.png}
\end{frame}

\begin{frame}{Colours}
   \begin{itemize}
   \item Perception: consider (maximising) distinctions in hue and brightness
     \end{itemize}
   \includegraphics[width=.8\linewidth]{colour-scales.png}
   \figcred{Introduction to the viridis color maps, Bob Rudis, Noam Ross and Simon Garnier}
\end{frame}

\begin{frame}{Colours}
   \begin{itemize}
   \item Perception: consider (maximising) distinctions in hue and brightness
     \item Consider colourblind viewers / black-and-white prints \url{https://cran.r-project.org/web/packages/viridis/vignettes/intro-to-viridis.html}
     \end{itemize}
   \includegraphics[width=.8\linewidth]{viridis-scales.png}
   \figcred{Introduction to the viridis color maps, Bob Rudis, Noam Ross and Simon Garnier}
\end{frame}

\begin{frame}{Exercise}
  \begin{itemize}
  \item What's going on here?
    \end{itemize}
  \includegraphics[width=.4\linewidth]{ggplot-plant.png}   \includegraphics[width=.4\linewidth]{new-phyt-cover.jpg}
\end{frame}

\begin{frame}{Summaries}
    \begin{itemize}
    \item Match scales, colours, fonts (basically, consistent style)
    \item Have information occupy most of the plot
    \item Plot as close to individual datapoints as possible
    \item Every approximation (especially smoothing) needs explicit mention
    \item Define all degrees of freedom (only) in figure caption / oral commentary
    \item Consider accessibility
    \end{itemize}
\end{frame}

\begin{frame}{Summaries}
  \begin{columns}
    \begin{column}{0.5\textwidth}
    \begin{itemize}
      \item \textcolor{red}{Plot command} + \textcolor{blue}{data} + \textcolor{brown}{mapping/aesthetic} + \textcolor{purple}{geoms} + \textcolor{gray}{styling options}
      \item Consider PNG and SVG(lite)
      \item Use the internet!
        \item Be creative, transparent, and thoughtful!
    \end{itemize}
    \end{column}
      \begin{column}{0.5\textwidth}
        \includegraphics[width=\linewidth]{ggplot-rw-ex.png}
      \end{column}
  \end{columns}
  \centering
  \includegraphics[width=.8\linewidth]{ggplot-rw2.png}
\end{frame}



\end{document}
